\section{Parameter setting}

The two generally accepted ways to get parameters into a program are
via command line arguments and via a config file.  Both these would be
cumbersome in tracter because there are a lot of individual modules
that need to know about parameters.  Rather, tracter uses environment
variables to set parameters.  The advantage is that the wrapper
program does not need to care about passing large numbers of
parameters to the right module.  The disadvantages are, well, none
that I've found.  Danil doesn't like it, and pointed out that multiple
threads wouldn't be able to distinguish parameters.  It does require a
slightly different programming mentality.  For instance, to set a
single parameter, rather than writing
\begin{verbatim}
tracter variable1=value1
\end{verbatim}
you can write
\begin{verbatim}
variable1=value1 tracter
\end{verbatim}
which isn't so different.
In the case of a whole configuration file, rather than writing
\begin{verbatim}
tracter -C config.txt
\end{verbatim}
you write
\begin{verbatim}
source config.sh
tracter
\end{verbatim}

What this tends to amount to is that it's convenient to call tracter
executables from scripts.  Typically, one might write
\begin{verbatim}
# Configuration
export Normalise_Endian=BIG
export Cepstrum_NCepstra=12

# Call the executable
tracter
\end{verbatim}

To aid this process, the tracter framework will dump every environment
variable requested in the form of a script.  This is currently done
with a command line argument
\begin{verbatim}
tracter -c
\end{verbatim}
producing something like
\begin{verbatim}
export FileSource_SampleFreq=8000.000000          # Default
export Normalise_Endian=BIG                       # Environment
export ZeroFilter_Zero=0.970000                   # Default
export Periodogram_FrameSize=256                  # Default
export Periodogram_FramePeriod=80                 # Default
export Noise_NInit=10                             # Default
export SpectralSubtract_Alpha=1.000000            # Default
export SpectralSubtract_Beta=0.000000             # Default
export MelFilter_NBins=23                         # Default
export MelFilter_LoHertz=0.000000                 # Default
export MelFilter_HiHertz=4000.000000              # Default
export MelFilter_LoWarp=0.000000                  # Default
export MelFilter_HiWarp=3400.000000               # Default
export MelFilter_Alpha=1.000000                   # Default
export Cepstrum_C0=1                              # Default
export Cepstrum_NCepstra=12                       # Environment
export Mean_Type=ADAPTIVE                         # Default
\end{verbatim}

Notice the comment.  It indicates whether the value came from a
compiled in default or from an environment variable.

Internally, environment variables are requested via the GetEnv()
method.  GetEnv() appends the requested string using an underscore to
the name of the object stored in mObjectName.  This is set via the
constructor and defaults to the name of the class.  If a different
name is supplied to the constructor it will be used.  This is useful
when the same module is used for two different purposes.  For
instance, to use a different window size for delta and acceleration
features, the code could read
\begin{verbatim}
Delta* d  = new Delta(x, "Delta1");
Delta* dd = new Delta(d, "Delta2");
\end{verbatim}
with window size set using
\begin{verbatim}
export Delta1_Theta=2  # Delta window 5
export Delta2_Theta=3  # Acceleration window 7
\end{verbatim}


%%% Local Variables: 
%%% mode: latex
%%% TeX-master: "tracter"
%%% End: 
