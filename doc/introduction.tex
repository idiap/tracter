\section{Introduction}

Tracter is a data-flow framework for signal processing.  It defines a
library of components that each typically do a small amount of
computation, but can become nodes in a directed graph where they work
together (although independently) to do something more useful.

Although tracter contains several implementations of basic algorithms,
it has developed into a wrapper for libraries of basic algorithms.

The components generally run serially, not in parallel; tracter does
not do concurrent data-flow in the sense of Kahn process networks
\citep{Lee1995}.

Tracter distinguishes sources and sinks.

%\begin{tikzpicture}[start chain]
%  \node [draw,on chain] {A};
%  \node [draw,on chain] {B};
%  \node [draw,on chain] {C};
%\end{tikzpicture}


%%% Local Variables: 
%%% mode: latex
%%% TeX-master: "tracter"
%%% TeX-PDF-mode: t
%%% End: 
